\documentclass[runningheads,envcountsame,a4paper]{llncs}
\pdfpagesattr{/CropBox [92 112 523 778]}

\usepackage[utf8]{inputenc}
\usepackage{microtype}
\usepackage{amsmath}
\usepackage{amssymb}
\usepackage{float}
\usepackage[algoruled,linesnumbered,noend]{algorithm2e}
\usepackage{enumitem}
\usepackage{cite}
\usepackage{multirow}
\usepackage{url}
\usepackage{xspace}
\usepackage{graphicx}

\newcommand{\coll}[1]{\url{#1}}

% REMOVE BEFORE SUBMISSION - begin part
\usepackage{verbatim}
\usepackage{textcomp}\usepackage{xcolor}
\newcommand{\notaestesa}[2]{%
 \marginpar{\color{red!75!black}\textbf{\texttimes}}%
 {\color{red!75!black}%
 [\,\textbullet\,\textsf{\textbf{#1:}} %
 \textsf{\footnotesize#2}\,\textbullet\,]}%
}
\newcommand{\YP}[1]{\notaestesa{YP}{#1}}
% \usepackage{tikz}
% \usetikzlibrary{shapes,arrows}
% \usetikzlibrary{snakes}
% \usetikzlibrary{positioning,patterns}


\newcommand{\mathfrc}[1]{\text{\fontfamily{frc}\selectfont#1}}

\usepackage{booktabs}
\newcommand{\otoprule}{\midrule[\heavyrulewidth]}
\newcommand{\lmidrule}{\midrule[.4\heavyrulewidth]}
\usepackage{tabularx}
\newcolumntype{T}[1]{>{\tsize} #1}
\newcolumntype{W}{>{\raggedleft\arraybackslash}X}
\newcolumntype{C}{>{\centering\arraybackslash}X}
\usepackage{array}


\newcommand{\inputdata}[1]{\noindent \emph{Input: }#1\\*}
\newcommand{\outputdata}[1]{\noindent \emph{Output: }#1\\}
\newcommand{\etal}{\textit{et al.}\xspace}
\newcommand{\ie}{\textit{i.e.}\xspace}
\newcommand{\st}{s.t.\xspace}
\newcommand{\wrt}{w.r.t.\xspace}
\renewcommand{\l}{\ensuremath{\ell}}
\renewcommand{\emptyset}{\ensuremath{\varnothing}}
\newcommand{\rev}{\ensuremath{\textrm{rev}}}
\newcommand{\SA}{\ensuremath{\textit{SA}}}
\newcommand{\Occ}{\ensuremath{\textit{Occ}}}


\begin{document}

\title{Assembling Overlap Graphs via Lightweight BWT Construction}

\author{%
  Paola Bonizzoni \and
  Gianluca Della Vedova \and
  Yuri Pirola \and
  Marco Previtali \and
  Raffaella Rizzi
}
\authorrunning{Bonizzoni \etal}

% \institute{DISCo,
%   Univ. Milano-Bicocca,
%   Milan, Italy\\ \email{\{beretta,bonizzoni,rizzi\}@disco.unimib.it}
% \and Dip. Statistica, Univ. Milano-Bicocca,
%   Milan, Italy\\ \email{gianluca.dellavedova@unimib.it}
% }

\maketitle

\YP{Rivedere titolo}

\YP{Page limit: 10 pages + a cleary marked appendix}

\begin{abstract}
In this paper we tackle the problem of reducing memory usage in sequence
assembly, providing an external memory algorithm to compute the overlap graph of
a set of sequence, which is the bottleneck in some sequencing approaches.

Our algorithm builds upon some recent results on lightweight Burrows-Wheeler
transform (BWT) construction and on bidirectional BWT to compute efficiently the
pairwise overlap of a set of sequences and to represent those overlaps with a
directed graph, with an overall time complexity that is the same as the known
algorithm to compute the BWT.
\end{abstract}

\begin{comment}
\section*{Legenda di supporto per la correzione}
\begin{itemize}
\item $n$: cardinality of the collection $R$
\item $\alpha$: extension string between overlapping reads
\item $w$: number of extensions of a BWT interval with some $\sigma$
\item \emph{length} of a $Q$-interval: length of $Q$
\item \emph{width} of a $Q$-interval $[b,e)$: difference $(e-b)$
\item $\$$-width $w_{\$}$: width of the $Q'\$$-interval inside a $Q'$-interval that is a prefix-interval (old \emph{dimension})
\item \emph{length} of an extension pair: $|S|$ + $|E|$ + $1$ (the old \emph{dimension})
\item $\mathcal{P}$: list of the prefix-intervals computed at each iteration (the old $\mathcal{T}$)
\item $C(\sigma)$: FM-index function (the old $C(\sigma, R)$)
\item $Occ(\sigma, i)$: FM-index function (the old $Occ(\sigma, i, R)$)
\item prefix-interval: $rev(Q)$-interval (on $B'$) representing a string $Q$ occurring as a prefix (the old $P$-interval denotes in this framework an interval representing a string $P$)
\end{itemize}
\end{comment}

\section{Introduction}
De novo sequence assembly is a fundamental step
in analyzing data from next generation sequencing technologies (NGS).
NGS technologies produce, from a given (genomic or transcriptomic) sequence, a huge amounts
of short sequences, called reads -- the most widely used current technology
produces $10^{9}$ reads with mean length $150$.
The large majority of the available
assemblers~\cite{Zerbino2008,Simpson2009,Peng2010,bankevich2012spades} are built
upon the notion of de Bruijn graphs where each $k$-mer is a vertex and an arc
connect two $k$-mers that have a $k-1$ overlap in some input read.
The main drawback of this approach is the amount of RAM needed, since storing
the vertices alone requires $\approx 15$ Gb when we want to assemble the human genome
and $k=27$.

For this reason alternative approaches have been developed recently, among those
the idea of string graph, initially proposed by Gene Myers~\cite{Myers2005}
before the advent of NGS technologies and further
developed~\cite{Simpson2010,Simpson2012} to incorporate some advances in text
indexing, such as the FM-index~\cite{Ferragina2005}.

The method builds an overlap graph whose vertices are the reads and where an arc
connects two reads that have a sufficiently large overlap.
For the purpose of assembling a genome some arcs might be uninformative.
In fact an arc $(r_{1}, r_{2})$ is called \emph{transitive} (or reducible) if
there its removal does not change the strings that we can infer from the graph.
It is immediate to notice that the final assembly is not influenced by the
presence of transitive arcs, which are therefore discarded to improve memory usage.
The final graph where all transitive arcs are removed is called the \emph{string} graph.
A main contribution of
Simpson and Durbin~\cite{Simpson2010} is (under the assumption
that there are not two reads that are one a substring of the other) a
characterization of transitive arcs, leading to an algorithm to construct a string
graph by outputting directly irreducible (\ie not transitive) arcs, skipping the
construction of an intermediate overlap graph, therefore significantly shrinking
memory requirements with linear time complexity.
An open problem of~\cite{Simpson2010} is to further reduce
the space requirements by developing an external memory algorithm to compute the
string graph, since their approach requires to keep in main memory the entire
BWT and the FM-index of all input data (\ie on the concatenation of all reads).

In the meantime, an investigation of external memory construction of the
Burrows-Wheeler Transform (BWT) and of related text indices (such as the
FM-index) has sprung~\cite{Bauer2011,Bauer2013,Ferragina2012} showing that it is possible to
arbitrarily reduce the use of RAM for building the BWT and indexing texts.
In this paper, based upon the work in~\cite{Bauer2011}, we show an external
memory algorithm to compute the overlap graph and an efficient, albeit not
entirely external memory) approach to remove all transitive arcs of the overlap
graph, hence resulting in a string graph.
% The construction of the String graph in~\cite{Simpson2010} works in two
% steps.
% In the first one, for each read $r$ in the collection $R$, the position
% in the BWT ($Q$-interval) of reads that share with $r$ an overlap given
% by a string $Q$ are computed in time linear in the size of $r$.
% In the second step, $Q$-intervals are extended to discover irreducible
% edges.
% Both steps strictly require to keep the whole FM-index and BWT sequence
% for $R$ and for the collection of reversed reads in main memory since
% the $Q$-intervals and the relative extensions cover different positions
% of the whole BWT.
% Moreover, the algorithm requires to recompute $Q$-intervals a number of
% times that is equal to the number of different reads in $R$ share the
% string $Q$ as a suffix.
%
%
In our approach, we keep in main memory the overlap graph, a portion (arbitrarily small)
of the BWT of the reads and some auxiliary arrays, while the entire BWT of the reads is stored on disk.
Since the overlap graph will be stored in RAM, the approaches presented in the
literature to remove the transitive arcs can be applied.


More precisely, we combine the recent work on bidirectional BWT that allows to
construct and query at the same time the BWT of a text and the BWT of the
reverse of a text~\cite{Lam2009} with that on external memory algorithm for BWT
construction~\cite{Bauer2011} to design an external memory BWT-based algorithm
that finds all occurrences of a pattern in a collection of texts by extending
the pattern incrementally by adding a character to the beginning or the end of
the pattern (while the standard pattern matching algorithm based on the FM-index
only extends the pattern by adding a character to the beginning).
This algorithm is the main component of our procedure to construct the string graph.

Notice that all large-size intermediate data are stored as external files that
can be read or written sequentially only.
Moreover, we strive to minimize the number of passes over those files, as a
simpler adaptation of the algorithm of~\cite{Bauer2011} would require a number
of passes equal to the number of input reads in the worst case.
The time complexity of our algorithm is $O(l \cdot  sort(n))$, which is also the
time complexity of the BCR algorithm~\cite{Bauer2013} for the external memory
construction of the BWT.
We also point out that in practice all input reads have approximately the same
length, hence the time to read the input data is $\Omega(ln)$, therefore our
time complexity differs from a lower bound only by a $\log n$ factor.








\begin{comment}
--------- non so dove vada-----------------


A key observation in our work is the fact that due to the fact that a
BWT consists of the symbols that precedes the lexicographic ordering of
suffixes of the reads of a collection, it is possible to "walk"
consecutively on the BWT of the collection of reads and of the reversed
version of the reads and then building a forward and backward extension
of a $Q$-substring by symbols of the alphabet lexicographically ordered.
All this can be done entirely using files for each symbol of the
alphabet and files for the BWT and its reversed version.
\end{comment}
% and by using the RAM only to keep an indexed representation of the
% string graph


\begin{comment}
descrizione The FM-index .... [inserire citazione FM-index]

With the advent of NGS data, the investigation of the BWT has moved
towards its use in indexing huge collections of texts, represented by
the reads of variable length.
The notion of Extended Borrows Wheeler has been proposed to deal with a
collection of texts~\cite{Bauer2011} and its potentiality in
Bioinformatics to index read data has been investigated in ....

While the BWT has been deeply used for the alignment of reads to a
reference genome, for the first time the BWT has been explored as a tool
for de novo assembly.

In~\cite{Simpson2010} Durbin addresses the problem of having a more
efficient space FM-index implementation of the string graph problem.

---results----

\end{comment}


\section{Preliminaries}
Let $\Sigma = \{\sigma_1, \dots, \sigma_m\}$ be an ordered finite
alphabet and let $r = a_1 \dots a_l$ be a string over $\Sigma$.
We denote by $r[i]$ the $i$-th symbol $a_i$ of $r$ and by $r[i,j]$ (with
$i \leq j$) the substring $a_i \dots a_j$ of $r$.
The length $l$ of $r$ is denoted by $|r|$.
The \emph{reverse} of $r$, denoted by $r'$ or $\rev(r)$, is the string
$a_l \cdots a_1$ obtained by reading $r$ from right to left.
The \emph{suffix} and \emph{prefix} of length $k$ of string $r$ are the
substrings $r[l-k +1, l]$ and $r[1, k]$ of $r$, respectively.
The $i$-suffix of $r$ is the suffix starting in position $i$ of $r$,
that is the substring $r[i, l]$.

Let $R = \{r_1, \cdots, r_n\}$ be a collection of $n$ strings over
$\Sigma$, where each sequence $r_i$ ends with a sentinel symbol \$
that is not in the alphabet $\Sigma$ and is considered lower than
any other symbol in $\Sigma$.
\YP{C'è inconsistenza sull'alfabeto. \$ appartiene o meno a $\Sigma$?} \notaestesa{RR}{Togliere \$ da $\Sigma$}
The \emph{generalized suffix array (GSA)} of the collection $R$ is a data
structure that allows to index the collection~\cite{Shi1996}.
It is defined as the array $\SA(R)$ where each element $\SA(R)[i]$ is
equal to $(k, j)$ if and only if the $k$-suffix of string $r_{j}$ is the
$i$-th lowest element in the lexicographic order of the set of all the
suffixes of the strings in $R$.
Clearly, the size of the suffix array is the sum of the length of all
strings in the collection $R$, that is $\sum_{r \in R}|r|$.

The \emph{Burrows-Wheeler transform (BWT)} of the collection $R$ is
defined as the sequence $B$ such that $B[i]=r_{j}[k -1]$ if $k > 1$
and $\SA(R)[i] = (k,j)$, otherwise $B[i]= \$$.
Informally, $B[i]$ is the symbol that precedes the $k$-suffix of
string $r_j$ where such suffix is the $i$-th lowest in the ordering
given by the generalized suffix array $\SA$.
Since $B$ is a string (actually, a permutation of the concatenation of
the strings in $R$), we are also able to define the
FM-index~\cite{Ferragina2005} of the BWT of the collection $R$.
The arrays that constitute the FM-index are defined as usual: $C(\sigma)$ denotes the number of symbols in $R$ smaller than the symbol
$\sigma$, while $\Occ(\sigma, i)$ is the array that reports the number
of occurrences in $B[1, i]$ of the symbol $\sigma$.
Notice that $\sigma$ may be the sentinel $\$$.

%Now, the string $B[i,k]$ for indexes $1 \leq i \leq k \leq N$ reports
%the sequence of symbols that precedes the ordered suffices in the
%generalized  suffix array $SA[i,k]$.

The two arrays $C(\sigma)$ and $Occ(\sigma, i)$
allow to compute in linear time
\YP{Giusto?}
all occurrences of a string $Q$ in the collection $R$
using the backward extension algorithm~\cite{Ferragina2005}.

% Observe that the string $B[i,k]$ for indexes $1 \leq i \leq k \leq N$
% reports the sequence of symbols that precedes the ordered suffices in
% the generalized suffix array $SA[i,k]$.






\subsection{Forward and backward extension of $Q$-intervals}

In the following let $B$ be the BWT for the collection $R$.
%\notaestesa{RR}{La definizione di $B'$ \'e stata spostata ad appena prima la definizione di linked intervals} Moreover, let $B'$ be the BWT of the set $R'$ of the reversed reads,
%that is $R' = \{ rev(r): r \in R\}$.
We are especially interested into some substrings or \emph{intervals} of $B$ that are identified by a string $Q$.
%In this case we will say that the \emph{$Q$-interval} of $B$ is the substring $B[b, e -1]$
%consisting of the symbols that precede the string $Q$ in any string of $R$.
%Equivalently the
A $Q$-interval is equal to the maximal interval $[b, e)$, such that
$Q$ is a prefix of all suffixes in the interval $SA[b, e-1]$.
We define the \emph{length} of a $Q$-interval as the length of
the string $Q$, and the \emph{width} of a $Q$-interval $[b, e)$ as the
difference $(e-b)$.
A $Q$-interval on $B$ of width larger than $1$ represents a string $Q$
that occurs more than once in $R$.
%Whenever $Q$ begins with a $\$$, the $Q$-interval is also called a \emph{$\$$-interval}.
%In the following, given an interval $q=[b,e)$, we denote with $R(q)=R(b,e)$ the reads containing the
%BWT symbols of $B[b, e-1]$.

The pattern matching algorithm using the FM-index is based on the idea of backward extension~\cite{Ferragina2005},
and it can be stated as a progressive refinement of a $Q$-interval by adding a new symbol to the beginning of $Q$.
More formally, given a $Q$-interval $[b,e)$ and a symbol $\sigma \in \Sigma$, the backward
extension of $[b,e)$ with $\sigma$ (or \emph{backward
$\sigma$-extension}) is the $\sigma Q$-interval $[b_{\sigma},e_{\sigma})$
-- the $\sigma Q$-interval represents the  suffixes starting with
the string $\sigma Q$. The interval $[b_{\sigma},e_{\sigma})$ can be quickly computed
from $[b,e)$ using the FM-index arrays $C(\sigma)$ and
$Occ(\sigma, i)$, since the width $(e_{\sigma}-b_{\sigma})$ of the $\sigma Q$-interval is equal
to the number of occurrences of  $\sigma$ in $B[b,e-1]$, that is $Occ(\sigma, e-1)-Occ(\sigma, b-1)$. Then $b_{\sigma} =C(\sigma) + Occ(\sigma, b-1)$, while $e_{\sigma} =C(\sigma) + Occ(\sigma, e-1) +1 $ \notaestesa{RR}{Controllare +1}.

In the following,
we say that a  $Q$-interval $[b,e)$, with a backward $\sigma$-extension $[b_{\sigma},e_{\sigma})$, has $(e_{\sigma}-b_{\sigma})$ possible backward
extensions with $\sigma$.
Given a $Q$-interval $q=[b,e)$ we denote with $X(q)$ (or $X(b,e)$)
the set of all  backward $\sigma$-extensions of $q$, that is $X(q)=\{ \sigma Q\text{-interval}: \sigma\in\Sigma\}$.
Similarly, the forward extension of a $Q$-interval $[b,e)$ with the symbol
$\sigma$ (or  \emph{forward $\sigma$-extension}) is the
$Q \sigma$-interval $[b^f_{\sigma},e^f_{\sigma})$ -- the forward extension represents the suffixes
starting with $Q \sigma$. In the following, we say that a
$Q$-interval $[b,e)$ with  a forward $\sigma$-extension
$[b^f_{\sigma},e^f_{\sigma})$ has $(e^f_{\sigma}-b^f_{\sigma})$
possible forward extensions with $\sigma$. Notice that the interval
$[b^f_{\sigma},e^f_{\sigma})$ is contained in $[b,e)$, but it is not as easy to compute as the corresponding backward extension. How to forward extend a BWT interval will be explained in the following relatively to intervals on $B'$, forward extended from the related linked intervals on $B$ (see Definition~\ref{def:linked-intervals}).

Let $B'$ be the BWT of the set $R'$ of the reversed reads,
that is $R' = \{ rev(r): r \in R\}$.

\begin{definition}
\label{def:linked-intervals}
Let  $Q=rev(Q')$.
Then the $Q$-interval $[b,e)$ on $B$ and the $Q'$-interval $[b',e')$ on $B'$ are
\emph{linked}.
\end{definition}

Clearly each $Q$-interval on $B$ is linked with exactly one $Q'$-interval on $B'$.
Moreover, two linked intervals have same width and length, hence $(e-b)=(e'-b')$~\cite{Lam2009,Simpson2010}.
Notice that when the length of the linked intervals is $1$ (\ie $Q=rev(Q)=\sigma$), then $[b,e)=[b',e')$,
since the two FM-index functions $C(\sigma)$ and $C'(\sigma)$ (related to $B'$) are the same.
In fact, the set of symbols of $R$ is equal to the set of symbols of $R'$.
It is possible to show that, given two linked intervals $[b,e)$ and $[b',e')$, even the backward $\sigma$-extension $[b_{\sigma},e_{\sigma})$ of $[b,e)$ and the forward $\sigma$-extension $[b^{'f}_{\sigma},e^{'f}_{\sigma})$ of $[b',e')$ are linked intervals.
In fact, since the interval $[b_{\sigma},e_{\sigma})$ is a $\sigma Q$-interval and the interval $[b^{'f}_{\sigma},e^{'f}_{\sigma})$ is a $rev(Q) \sigma$-interval, then they are linked by Definition~\ref{def:linked-intervals}.
Moreover, the number of possible backward extensions with $\sigma$ of $[b,e)$ is equal to the number of possible forward extensions with $\sigma$ of $[b',e')$.
Also,  the start $b^{'f}_{\sigma}$ is given by adding to $b'$ the number of symbols that are smaller than $\sigma$ and are in $B[b,e-1]$,
while the end $e^{'f}_{\sigma}$ is given by adding to $b^{'f}_{\sigma}$ the number of symbols $\sigma$ which are in $B[b,e-1]$~\cite{Simpson2010}.

\begin{proposition}
\label{proposition:linked-intervals}
Let $B$ and $B'$ be respectively the BWT of a text $T$ and $rev(T)$, let $Q$ be a substring of $T$ and let $\sigma$ be a symbol.
Let $[b,e)$ be the $Q$-interval on $B$.
Then it is possible to compute in $O(1)$ time the $\sigma Q$-interval on $B$ and the $rev(Q) \sigma$-interval on $B'$
using only of the FM-index functions $C(\sigma)$ and $Occ(\sigma, i)$.
\end{proposition}

A special case of this argument applies when
$Q$ is a prefix of $w > 0$ reads in $R$.
In this case, the $Q$-interval $[b,e)$ has $w$ possible backward $\alpha$-extensions, and  $B[b,e-1]$ contains $w$ $\$$s.
At the same time, the $\$ Q$-interval (\ie the backward $\$$-extension of $[b,e)$) represents the set of $w$ reads with prefix $Q$.
Hence the $rev(Q)$-interval $[b',e')$ (linked to $[b,e)$) has $w$ forward $\$$-extensions.
Finally, since $\$$ is the smallest symbol of $\Sigma$ \notaestesa{RR}{Risolvere la questione $\$$ incluso in $\Sigma$}, the $rev(Q) \$$-interval is $[b',b'+w)$.


%When a substring $Q$ occurs $m>0$ times as a suffix in $R$, then the
%$rev(Q)$-interval on $B'$ contains $m>0$ symbols $\$$, and the $\$
%rev(Q)$-interval represents the set of the $m$ reads %where $Q$ occurs
%as a suffix.

%\begin{definition}
%A \emph{S-interval} is a $rev(Q)$-interval $[b',e')$ on $B'$ having
%$m>0$ backward extensions with symbol $\$$  ($m$ is called the
%\emph{dimension} of the \emph{S-interval}).
%\end{definition}

Our procedure will compute the arc set $A_r$ of all outgoing arcs for each read $r$,
by building all possible $Q$-intervals that might be overlaps of two
reads in $R$, and then iteratively forward extending only the $Q$-intervals that
are prefixes of some read $r_{e}$ until the sentinel symbol \$ is
found (\ie the read $r_{e}$ has been completely read).
In any instant, for each read $r$ only one extension $r_{e}$ is stored; in fact after examining a $Q$-interval, the read $r_{e}$ is update, if necessary.



Two parts of the algorithm exploit the FM-index to perform
backward extensions to determine the intervals of overlapping
prefixes and forward extensions to extend those prefixes and build the
terminal extensions.


\section{Building the Overlap Graph}



In order to simplify the algorithm we make two assumptions regarding the read set $R$.
The first assumption is that $R$ is made of \emph{substring-free} strings, that is there are not two reads $r_{1}, r_{2}\in R$ such that
$r_{1}$  is a substring of  $r_{2}$.
The second assumption is that an arc $(r_{1}, r_{2})$ of the String Graph exists only if the overlap of $r_{1}$ and $r_{2}$ is at least a certain constant $\tau$.



Observe that the construction of the String Graph in~\cite{Simpson2010} works in two
steps.
In the first one, for each read $r$ in the collection $R$ all
$Q$-intervals, where $Q$ is the overlap of $r$ and some read extending $r$,
are computed in time linear in
the  length of $r$.
In the second step, each of these intervals are forward extended until
we find the first \$; such a $Q\alpha\$$-interval becomes an
irreducible edge.
Observe that the algorithm  in~\cite{Simpson2010} requires to keep the whole FM-index
and BWT sequence for $R$ and $R'$ in main memory, since it is impossible to
restrict in a portion of the BWT or of the FM-index the $Q$-intervals
and their extensions.
Moreover, the algorithm requires to recompute the $Q$-intervals a number of
times that is equal to the number of different reads in $R$ share the
string $Q$ as a suffix.


In our approach, we keep the BWT $B$  and  $B'$, respectively  for the collection $R$ of reads and its
reverse version, $R'$,  in secondary memory and we read them sequentially,
and during this process we iteratively forward and backward extend each
$Q$-interval computed at a previous iteration.

Our algorithm  will compute the arc set $A$  of the overlap graph
by building all possible $Q$-intervals that might be overlaps of two
reads in $R$, and then iteratively forward extending only the $Q$-intervals that
are prefixes of some read $r$ until the sentinel symbol \$ is
found, meaning that the read $r$ has been completely processed by the algorithm.
More precisely,    all edges    $e=(r,r')$ of the overlap graph incident on $r$ where the overlap of $r$ and $r'$ is a fixed string $S$ (called {\em seed}) are  associated to a pair  $(q'_1, q'_2)$ of $Q$-intervals on $B'$,
where $q'_1$ is the interval  representing   read  $r$ and $q'_2$ is the  $S$-interval representing all reads
 $r'$ with overlap $S$ with read $r$. Such a pair is called {\em terminal extension pair} and is formally defined below.





We will also detail how we can compute efficiently all terminal extension pairs
using the idea of bidirectional FM-index~\cite{Lam2009}.

\begin{definition}\label{def:extension-pair}
Let $S$ and $E$ be two strings and let $L$ be equal to $|S|+|E|$.
Then let $q_{1}'$ be the $rev(E)rev(S)\$$-interval of $B'$ and let  $q_{2}'$ be the $\$rev(S)$-interval of $B'$.
If both $q_{1}'$ and  $q_{2}'$ are nonempty, then the pair  $(q_{1}', q_{2}')$ is called
an \emph{extension pair} of length $L$.
% is a pair $(q_{1}', q_{2}')$ of
% intervals on $B'$, such that $q_{2}''$ is a $rev(E)rev(S)\$$-interval,
% $q_{2}'$ is a $\$rev(S)$-interval, and $|rev(E)rev(S)| = L$.
Moreover $(q_{1}', q_{2}')$ is a \emph{terminal} extension pair
if and only if $q_{1}'$ has a nonempty \$-backward extension on $B'$.
\end{definition}

The interval $q_{1}'$ represents   the reads with prefix $SE$, while $q_{2}'$
represents the reads with suffix $S$.
The  string $E$ is called the \emph{extension} of the
pair, while the string $S$ is called the \emph{seed}.
In other words, the extension pair corresponds to the fact that the reads related to $q_{2}'$ are extended by the reads
of $q_{1}'$.
In a terminal extension pair, the concatenation of the seed and the extension is
equal to some read $r \in R$ and it is immediate to verify  that for each read $r'$ represented in the $q_{2}'$-interval, since $r'$ is extended by $r$,
the pair $(r', r)$ is an edge of the overlap graph.


The correctness of our approach relies on the fact that all possible
terminal extension pairs will be examined during the iterations and thus
the arc set of the final overlap  graph over $R$ will be computed.
More precisely, we will construct all extension pairs with seed at least $\tau$ long, since only such extension pairs can result in arcs.
Moreover, the extension pairs are computed for increasing dimension.






To compute the extension pairs we need to find, among all Q-intervals, those corresponding to prefixes of a read.
This fact is formalized with the notion of prefix-interval.

\begin{definition}
A \emph{prefix-interval} is a $rev(Q)$-interval $[b',e')$ on $B'$ containing
a $rev(Q)\$$-interval of width $w_{\$}>0$.
Moreover, $w_{\$}$ is called the \emph{\$-width}
of the prefix-interval.
\end{definition}

A \emph{prefix-interval} $[b',e')$ of width $w_{\$}$ represents the $(e'-b')$
reads sharing $rev(Q)$, and $w_{\$}$ among these reads share the string $Q$ as a
prefix.

The main step of our external memory algorithm consists in  building efficiently
the $Q$-intervals on $B$ and the $rev(Q)$-intervals on
$B'$ by iteratively increasing the length $l$ of $Q$.
The first iteration considers $Q$-intervals where $l=\tau$.
Each iteration increases $l$ by one, until $l$ is equal to the maximum length
of a read in $R$.


Our algorithm builds upon two methods presented in~\cite{Bauer2011,Cox2012}.
The first method computes the BWT of a collection of strings using $|\Sigma|$
external files $F_b$, where each file $F_b(\sigma)$ contains the portion of BWT related to the suffixes
starting with the symbol $\sigma$.
The second method allows to construct (in external files) $Q$-intervals
of increasing length. \notaestesa{RR}{Introdurre qui brevemente le procedure processInterval e processLinkedInterval}

\begin{comment}
In particular, the procedure \emph{processInterval} given
in~\cite{Cox2012} is used in our algorithm to produce the sorted list
(lexicographical order) of the intervals of a given length $l$, from the
sorted list of the intervals of length $l-1$ (intervals of length $1$
can be easily obtained by means of the FM-index function $C$).
\notaestesa{Raffa}{Togliere lo pseudocodice della procedura processInterval: \'e della Rosone e non nostro.}.
\notaestesa{GDV}{ Concordo di togliere lo pseudocodice.
Bisogna dire però cosa calcola la procedura}

We extend the procedure \emph{processInterval} to
\emph{processLinkedInterval}, where we produce not only the $Q$-intervals on $B$, but also the linked
$rev(Q)$-intervals on $B'$.
\end{comment}



Let us now describe the main algorithm  \emph{buildGraph$(R, \tau)$} that
builds the overlap graph.
At each iteration $j$ (starting from $j=0$) the procedure builds  two  lists
that are actually implemented by files that are read sequentially,

\begin{itemize}
\item $\mathcal{Q}_{j+1}$, containing the nonempty $Q$-intervals $[b, e)$ on $B$ with length $(j+ 1 + \tau)$,
each one with its linked
  $rev(Q)$-interval $[b', e')$ on $B'$. The list is  ordered by increasing values of $b$.
%\item $\mathcal{P}$, containing the nonempty $rev(Q)$-intervals $[b', e')$ of
 % length $(j+ 1 + \tau)$ which are prefix-intervals, each one with its \$-width $w_{\$}$. The list is ordered by increasing values of $b'$.
\item $\mathcal{E}_{j +1}$, containing the extension pairs $(q_{1}', q_{2}')$ of
  length $(j+1 +\tau)$.  The list is ordered by increasing values of $b'$.

  %, each one with its extension $e$ $> 0$
\end{itemize}

A $Q$-interval  $[b,e)$ in $\mathcal{Q}_{j+1}$ represents a
string $Q$, of length $(j+1 +\tau)$, while
the linked $rev(Q)$-interval  $[b',e')$ in $\mathcal{Q}_{j+1}$  represents a
string $Q$, of length $(j+1+ \tau)$, occurring $w_\$$ times as a prefix in $R$.
In fact,   the interval $[b',b'+w_\$)$ represents the reads having $Q$ as
a prefix. An extension pair $(q_{1}', q_{2}')$ in $\mathcal{E}_{j+1}$ represents the fact that the reads related to $q_{2}'$ are extended by the reads of $q_{1}'$ and the overlap (\ie the seed) is of length $<(j+1+\tau)$.
Notice that the lists $\mathcal{Q}_{j+1}$ and  $\mathcal{E}_{j+1}$ are stored in $|\Sigma|$ external file.

Now,  extension pairs in $\mathcal{E}_{j+1}$ are computed from extension pairs in  $\mathcal{E}_{j}$ or from prefix-intervals in the list $Q_j$ that may provide seeds of length $j + \tau$ of extension pairs. In order to discover new seeds from prefix-intervals, the algorithm at iteration $j$  builds the list $\mathcal{P}$, called {\em seed-list},  of prefix-intervals  which are $rev(Q)$-intervals $[b', e')$ in the list $Q_j$. Observe that  $\mathcal{P}$  which is  stored in a unique external file.



%\notaestesa{paola}{since they are lexicographically ordered w.r.t. .... dire come sono ordinati per simbolo iniziale}

%At each iteration, we analyze those list to find new possible arcs of the String Graph and to determine whether they are reducible.
%Clearly,  the lists obtained with the  $j$-th iteration will be used in the subsequent  $(j+1)$-th iteration.




Before the first iteration of the algorithm , the list $\mathcal{Q}_0$ contains the
$Q$-intervals of length $\tau$ (representing strings $Q$ of length $\tau$), while  $\mathcal{E}_0$ and $\mathcal{P}$ are empty.
The list $\mathcal{Q}_0$ is computed  by building intervals of increasing length by means of our procedure \emph{processLinkedInterval} (starting from intervals of length $1$, easily computed from $C(\sigma)$).



During the $j$-th iteration, we compute the lists described before as follows:
(i) the list $\mathcal{Q}_{j+1}$ of the
intervals of length $(j+\tau+1)$  -- by backward extending, with each symbol different from \$, each $Q$-interval $[b, e)$ in $\mathcal{Q}_{j}$ (and forward extending, as detailed  below, the
linked $rev(Q)$-interval on $B'$). While computing $\mathcal{Q}_{j+1}$ we compute the seed-list $\mathcal{P}$.
We compute
(ii) the list $\mathcal{E}_{j+1}$ of
extension pairs of length $(j+\tau+1)$  taking ordered intervals   either from the  list   $\mathcal{E}_{j}$ or from the seed-list $\mathcal{P}$.    In the first case, a new  extension pair
$(q_{1}',q_{2}')$ is added to $\mathcal{E}_{j+1}$, since   $q_{1}'$ is a backward $\sigma$-extension
of an interval $[b',b'+w_\$)$ which is the forward \$-extension of a prefix-interval in $\mathcal{P}$,  -- the string $E$ of the computed pair consists of a single character. In the second case,
we obtain  extension pairs $(q_{1}',q_{2}') \in \mathcal{E}_{j+1}$  from $(q',q_{2}')\in \mathcal{E}_{j}$, where
$q_{1}'$ is a backward $\sigma$-extension
of a $q'$   -- the string $E$ of the computed pair is one
character longer than that one of $(q', q_{2}')$ computed in iteration $j$.
% --
% or (iii c)
% $q_{2}'$ is a $\sigma$ backward extension
% of $q'$ where $(q_{1}',q')\in \mathcal{E}_{j}$ -- we move the first character of
% the extension to the end of the seed.


% The list $\mathcal{Q}_{j}$ will be used during the next iteration
% $(j+1)$, and represents common strings of length increased by $1$.
\noindent
{\bf  Computing  the list $Q_{j+1} $ and $Q$-intervals}

Since the core of our approach is the extension of $Q$-intervals, we will
describe the steps that are necessary to achieve that.
The problem has been tackled in the literature by
the procedure \emph{processInterval}~\cite{Cox2012} which is also used in our algorithm to produce the sorted list
(lexicographical order) of the intervals of a given length $l$, from the
sorted list of the intervals of length $l-1$. Notice that the intervals of length $1$
can be easily obtained from the  function $C(\sigma$) which is part of the FM-index.

The procedure \emph{processInterval} takes in input a $Q$-interval $[b,e)$ of
length $l$, the BWT $B$, a vector $\Pi$ of $|\Sigma|$ counters, an array $F$ of
$|\Sigma|$ external files, and writes all the backward $\sigma_i$-extensions of
$[b,e)$ into the files $F$. Notice that is never necessary computing $Q$-intervals, where $Q$ starts with \$, since we are interested in strings $Q$ composed of symbols $\sigma$ different from \$.
More precisely, the $\sigma_i Q$-interval is appended to the file $F[i]$.
This procedure computes the FM-index function $Occ(\sigma_i, i)$ using two
vectors: the (global) input vector $\Pi$ and the local vector $\pi$.
The first one stores in $\Pi[i]$ the number of occurrences of  the symbol
$\sigma_i$ in the BWT prefix $B[1,b-1]$ (\ie $Occ(\sigma_{i}, b-1)$) while $\pi[i]$ stores the number of occurrences of $\sigma_i$ in the interval $[b,e)$. At the beginning, the vector $\Pi$ contains $|\Sigma|$ counters, each one set to $0$, and the $|\Sigma|$ files $F$ are empty.
After calling this procedure on each one of the $Q$-intervals of length $l$, supposed to be available in lexicographic order (thus, two consecutive intervals are disjoint), then  at the end each file $F[i]$ will contain the sorted list of the intervals of length $(l+1)$, starting with symbol $\sigma_i$. Considering files $F$ in the order $F[1], F[2], \ldots ,F[|\Sigma|]$, we obtain the sorted list of the intervals of length $(l+1)$. After each procedure call, the vector $\Pi$ is updated, to be passed to the next procedure call.\\
By assuming that input $Q$-intervals of length $l$ are available in lexicographical order, this procedure is able to update $\Pi$ and $\pi$ by reading the symbols of $B$ consecutively.
In fact, the $Q$-intervals occurs consecutively along the BWT $B$ and  thus while reading $B$ the number of symbols needed to update $\Pi$ and $\pi$ are counted.

\noindent
{\bf Computing  linked $Q$-intervals on $B'$.}

We have extended \emph{processInterval} to manage at the same time  the linked intervals on
$B'$; see procedure \emph{processLinkedInterval}.
Our procedure takes in input a $Q$-interval $[b,e)$ and its linked
$rev(Q)$-interval $[b',e')$, and outputs (appending to the external files $F$)
the extended $\sigma_i Q$-intervals, each  with the linked $rev(Q)
\sigma_i$-interval on $B'$
Our procedure writes to $F$ the start and the end of the output intervals on the
complete BWTs $B$ and $B'$ (see line 5), differently from~\cite{Cox2012} where they write, for each output $Q$-interval, its start \wrt to the BWT segment related to the starting symbol of $Q$, and its width.
At each iteration $i$ over the alphabet symbols the forward $\sigma_i$-extension
of the $rev(Q)$-interval $[b',e')$ is performed by counting
the number of symbols in $B[b,e)$ which are lexicographically smaller than
$\sigma_i$ just as in~\cite{Simpson2010}  \notaestesa{paola}{qua a mio parere va detto come viene fatto}.
We recall that the correctness of \emph{processLinkedInterval} is based on the
fact that all $Q$-intervals are processed (and extended) in lexicographic order.

% \notaestesa{Raffa}{Qui invece va la descrizione della nuova procedura \emph{processSingleInterval}, sempre se la teniamo... che determina la lista di tutti gli intervalli estesi di un $Q'$-interval su $B'$ - vedi modifica 6}.
% Notice that the files $F(\sigma)$ must be read according to the lexicographic order.







\begin{comment}
More precisely, the steps .... in algorithm \emph{processLinkedInterval}
directly provides the values of the new $\sigma Q$-interval and such
value is appended at the end of the new updated list (or file), in order
to maintain the lexicographic ordering.
In fact, the sorted property of the file, allows to update the files
while reading the files themselves, thus minimizing the operations on the
file.
\notaestesa{PB}{nota bene il passo dal 3 al 9 non si capisce come sia
implementato in termini di memoria esterna ed interna - io sarei per
descrivere la procedura processLinkedInterval- dettagliandola rispetto
ai linked intervals, che però vanno definiti prima, in termini
costruttivi}.


Intervals $[b,e)$ in $\mathcal{Q}_j$ represent common strings $Q$ of
length $(j+\tau)$.
Their link to the $rev(Q)$-interval $[b',e')$ on $B'$ is easily
maintained (during the backward extension of the intervals of length
$(j+\tau-1)$ to produce the intervals in $Q_j$) by Prop.~\ref{proposition:linked-intervals}.
This list is partioned into $|\Sigma|$ files according to the starting
symbol of $Q$, and maintained sorted by ascending value of the start
$b$.
\end{comment}

\notaestesa{RR}{Aggiungere tutti i riferimenti allo pseudocodice dopo averlo sistemato}

\noindent
{\bf  Computing the {\em seed-list}.}
Each element of the list $\mathcal{P}$ is made of a  \emph{prefix-interval} $[b',e')$
and its \$-width $w_\$ $ (recall that $w_\$ $ is  the number of     \$ backward \$-extension of  the $Q$-interval $[b,e)$ that is linked to $[b',e')$).
The list $\mathcal{P}$ is actually stored as a file and is sorted by ascending value of the starting
value $b'$ (\ie lexicographical order of $rev(Q)$). \notaestesa{RR}{Citare il classico problema di ordinamento in memoria esterna con la relativa citazione}.
%Each \emph{prefix-interval} $[b', e')$ with length $(j+  \tau)$ and  \$-width $w_\$$ is obtained directly from the previous step 2 as linked to a $Q$-interval $[b, e)$ on $B$ with exactly $w_\$$ %backward \$-extensions.
%, and is computed (at the iteration $j$) from the list $\mathcal{Q}_j$, by
%taking into account that a
%\emph{prefix-interval} $[b', e')$ with length $(j+\tau)$ and  \$-width $w_\$$ is linked to a $Q$-interval $[b, e)$ on $B$ with exactly $w_\$$ backward \$-extensions.
Notice that the only operations allowed on that file are appending an item to
the end and popping (\ie reading from the head and deleting) an item.
%Let us recall that a \emph{prefix-interval} $[b', e')$ correspond to
%the ordered list of reads having substring $rev(Q)$ on $B'$.

%in $B'$.

\noindent
{\bf Computing  extension pairs in $\mathcal{E}_{j+1}$ from the seed list.}


From each  prefix-interval in $\mathcal{P}$ we then compute the extension pairs $(q_{1}', q_{2}')$ with
length $(j+\tau+1)$ and extension equal to $1$ (\ie the string $E$ is composed of one symbol) as follows.
More precisely, given an element  $<[b',e'), w_\$>$ of $\mathcal{P}$,
we pose $q_{2}'$ equal to the forward \$-extension of  $[b',e')$, where   $q_{2}' = [b',b'+w_\$)$
(recall that $q_{2}'$ represents reads with suffix the seed $S$).


Then
for each $q_{1}'\in X(b',b'+w_\$)$ we add the extension pair  $(q_{1}',q_{2}')$ to
the list $\mathcal{E}_{j+1}$ (recall that $ X(b',b'+w_\$)$ is the list of backward extensions of interval $[b',b'+w_\$)$).


Notice that the interval $[b', b'+w_\$)$ (under our assumption) cannot have
extensions with symbol \$: in fact, no read in $R$ can be equal to the string $Q$.

\noindent
{\bf Computing  extension pairs in $\mathcal{E}_{j+1}$ by extending   extension pairs in $\mathcal{E}_{j}$.}

Moreover we try to extend each extension pair with length $(j+\tau)$ (and extension $>1$) into some
extension pairs of length $(j+\tau+1)$.
More precisely, for each extension pair $(q', q_{2}')$ in  $\mathcal{E}_j$ and
for each $q_{1}'\in X(q')$ we add the extension pair  $(q_{1}',q_{2}')$ to
the list $\mathcal{E}_{j+1}$.
% At each iteration $j$, the interval $q_{1}'$ of a pair $(q_{1}', q_{2}')$ with
% extension $e$ in $\mathcal{E}_j$ is backward extended on $B'$, where
% $q_{1}'$ is given.
% Observe that each $\sigma$  backward extension corresponds to a  forward
% extend by $\sigma$ on $B$ the  \$ $Q$-intervals, \ie computing the
% interval of reads having $Q \sigma$ has prefix (see figure \ref{}).
If  $q'$ has an  extension with symbol \$
into the interval $[b'_r, b'_r+1)$, then given  $r$ the unique  read represented by that
interval, for   each read  $r'$ represented by the
interval $q_{2}'$, the edge $(r,r')$ is added to the arc set of the overlap graph.
 In this last case, no extension pairs are added to $\mathcal{E}_{j+1}$ since they relative edges have been discovered.

%$j$-th iteration.
%Otherwise (if $q'$ has no extensions with symbol \$), an extension
%pair $(q_e, q_{2}')$ (of dimension $(j+\tau+1)$ and extension $e+1$) is
%added to the list $\mathcal{E}_{j+1}$, for each $q_e \in X(q_{1}')$, where
%$X(q_{1}')$ is the set of the intervals obtained by backward extending
%$q_{1}'$ with each $\sigma \in \Sigma$.

At each iteration $j$, the list $\mathcal{E}_{j+1}$ receives extension
pairs both from the seed-list  and $\mathcal{E}_{j}$.
Since, the list  $\mathcal{E}_{j}$  and the seed-list are ordered w.r.t. the start of  $q'_1$ and $Q$-intervals in $\mathcal{P}$, respectively and since the  length of $Q$-intervals $[b', b'+w_\$)$ in $\mathcal{P}$ is
equal to the length  of  $Q$-intervals $q'_1$ in $\mathcal{E}_{j}$, the
algorithm can extend the union of the two lists at each iteration, and therefore keep
sorted also the list $\mathcal{E}_{j+1}$, by ascending value of the
start of $q'$ on $B'$.\notaestesa{GDV}{ dove ho la
  garanzia che non ci sono salti di inizio di $q_{1}'$?} \notaestesa{RR}{Spiegare qui la questione degli extension pairs in $\mathcal{E}_j$, con diverso seed, che hanno in comune $q'$}




%A $rev(Q)$-interval  $[b',e')$ in $\mathcal{P}$ represents common
%strings $Q$ of length $(j+\tau)$ occurring $w_\$$ times as a prefix in $R$.
%Recall that the interval $[b',b'+w_\$)$ represents the reads having $Q$ as
%a prefix.


%The seeds of the extension pairs in $\mathcal{E}_j$
%%(partitioned in $|\Sigma|$ files according to the first symbol of ??)
%have a length $< (j+\tau)$ and are the common substrings considered at
%the previous iterations $< j$.



\section*{Pseudocode}
Denoto con $A(q_{1}', q_{2}')$ il set delle coppie tra i reads del secondo intervallo e i reads del primo intervallo.
Ricordarsi di indicizzare nella parte discorsiva i file e gli array FM-index per simbolo e non per indice.
Inoltre rinominerei processLinkedInterval in extendLinkedIntervals

\begin{algorithm}
\SetKwInOut{Input}{Input}\SetKwInOut{Output}{Output}

Let $R'$ be the set of the reverse of the input reads, let $B$ and $B'$ be respectively the BWT of $R$ and $R'$, let $C$ be the corresponding array of the FM-index, and let $\mathcal{Q}_{0}$ be the $Q$-intervals of length $\tau$, each with its linked $rev(Q)$-interval\;

Let $A$ be the arc set of the overlap graph, initially empty\;

$\mathcal{E}_0 \gets \emptyset$\;

$j \gets 0$\;

\tcp{Until $\mathcal{Q}_j$ or $\mathcal{E}_j$ is nonempty}
\While{$(\mathcal{Q}_j \cup \mathcal{E}_j) \ne \emptyset$}{

 $\mathcal{Q}_{j+1}\gets$ $|\Sigma|$ empty files\;

\tcp{Now we can compute and sort all prefix-intervals, and we compute all $Q$-intervals of length $(j+\tau+1)$}

$\mathcal{P}\gets$ empty file\;

$\Pi(\sigma) \gets 0$ for each $\sigma \in \Sigma$\;

 \ForEach{$\sigma \in \Sigma$}{
      \ForEach{$\langle [b,e),[b',e')\rangle \in \mathcal{Q}_j(\sigma)$}{
	processLinkedInterval$([b, e), [b', e'), \mathcal{Q}_{j+1})$\;
        \If{$[b',e')$ is a prefix-interval} {
             Let $w_\$$ be the number of \$-extensions of $[b,e)$\;
             Add $\langle [b', e'), w_\$ \rangle$ to $\mathcal{P}$\
}
}
  }

  sort $\mathcal{P}$\;

\tcp{Now we can compute all extension pairs of length $(j+\tau+1)$}

$\Pi(\sigma) \gets 0$ for each $\sigma \in \Sigma \cup \{\$\}$\;

 $\mathcal{E}_{j+1}\gets$ $(|\Sigma|+1)$ empty files\;

    $\langle [b', e'), w_\$ \rangle \gets$ the first prefix-interval of $\mathcal{P}$\;
    $(q_{1}', q_{2}') \gets$ the first extension pair of $\mathcal{E}_j$\;

\tcp{Until no more elements can be read from $\mathcal{P}$ and $\mathcal{E}_j$}
    \While{$((\mathcal{P} \cup \mathcal{E}_j) \neq \emptyset)$} {
		\If{$(b'$ is less than the start of $q_{1}'$)} {
\tcp{We extend a prefix-interval, and (if possible) compute an extension pair whose extension consists of a single character}
			extendPrefixInterval$([b', e'), w_\$, \mathcal{E}_{j+1})$\;
                        $\langle [b', e'), w_\$ \rangle \gets$ the next prefix-interval of $\mathcal{P}$\;
                      }
		\Else {
\tcp{We extend a preexisting extension pair and, if necessary, update the arc set of the overlapping graph}
      $A_t \gets$ extendExtensionPair$(q_{1}', q_{2}', \mathcal{E}_{j+1})$\;
                  \If{$A_t$ is not empty}{
                   Add $A_t$ to A\;
             }


    $(q_{1}', q_{2}') \gets$ the next extension pair of $\mathcal{E}_j$\;
 		}
	}

  $j \gets j+1$\;
}
\Return{the overlap graph $G(R,A)$}\;

\caption{buildGraph($R$, $\tau$)}
\label{alg:build-graph}

\end{algorithm}

\begin{algorithm}
\SetKwInOut{Input}{Input}\SetKwInOut{Output}{Output}
\tcp{$\Pi[\sigma]$ must be number of $\sigma$ in $B[0,b)$, for each $\sigma \in \Sigma$}
Update $\Pi$ such that $\Pi[\sigma]$ is the number of $\sigma$ in $B[0, b)$\;
$\pi(\sigma) \gets$ number of $\sigma$ in $B[b,e)$, for each $\sigma \in \Sigma$\;

$\pi sum \gets 0$\;
 \ForEach{$\sigma \in \Sigma$}{
   \If{ $\pi[\sigma] > 0$ } {
\tcp{We compute the the $\sigma Q$-interval on $B$ and the $rev(Q) \sigma$-interval on $B'$}
       $b_e \gets C[\sigma] + \Pi[\sigma]$+1\;
       $e_e \gets b_e + \pi[\sigma]$\;
       $b'_e \gets b' + \pi sum$\;
       $e'_e \gets b'_e + \pi[\sigma]$\;
      Append $\langle [b_e, e_e), [b'_e, e'_e)\rangle$ to $F(\sigma)$\;
     }
\tcp{$\pi sum$ maintains the number of symbols $< \sigma$ in $B[b,e)$}

     $\pi sum \gets \pi sum + \pi[\sigma]$\;
                      }
$\Pi \gets \Pi+\pi$\;

\caption{processLinkedInterval($[b,e)$,$[b',e')$ $F$)}
\label{alg:process-linked-interval}
\end{algorithm}

\begin{algorithm}
\SetKwInOut{Input}{Input}\SetKwInOut{Output}{Output}
\tcp{$\Pi[\sigma]$ must be number of $\sigma$ in $B'[0,b')$, for each $\sigma \in \Sigma \cup \{\$\}$}
Update $\Pi$ such that $\Pi[\sigma]$ is the number of $\sigma$ in $B'[0, b')$\;

$\pi(\sigma) \gets$ number of $\sigma$ in $B'[b',e')$, for each $\sigma \in \Sigma \cup \{\$\}$\;
$\pi_p(\sigma) \gets$ number of $\sigma$ in $B'[b',b'+w_\$)$, for each $\sigma \in \Sigma$\;

\tcp{We check if $[b',e')$ has nonempty backward \$-extension}
\If{ $\pi[\$] > 0$ } {
 % $b'_p \gets \Pi[ 0 ]$\;
   $q_{2}' \gets [(\Pi[\$]+1), (\Pi[ \$ ] + \pi[ \$ ]+1))$\;
\tcp{The interval $[b',b'+w_\$)$ must have (under our assumption) empty backward \$-extension}
 \ForEach{$\sigma \in \Sigma$}{
\tcp{We compute all backward $\sigma$-extensions of $[b',b'+w_\$)$}

    \If{ $\pi_p[ \sigma ] > 0$} {
        $q_{1}' \gets [(C(\sigma)+\Pi[\sigma]+1), (C(\sigma)+\Pi[ \sigma ] + \pi_p[ \sigma ]+1))$\;
      Append $(q_{1}', q_{2}')$ to $F(\sigma)$
    }
  }
}
$\Pi \gets \Pi + \pi$

\caption{extendPrefixInterval($[b', e')$, $w_\$$, $F$)}
\label{alg:processPInterval}
\end{algorithm}

\begin{algorithm}
\SetKwInOut{Input}{Input}\SetKwInOut{Output}{Output}
Let $b'$ and $e'$ be respectively the start and the end of $q_{1}'$\;
%Let $b_{s}'$ and $e_{s}'$ be respectively the start and the end of $q_{2}'$\;

\tcp{$\Pi[\sigma]$ must be number of $\sigma$ in $B'[0,b')$, for each $\sigma \in \Sigma \cup \{\$\}$}
Update $\Pi$ such that $\Pi[\sigma]$ is the number of $\sigma$ in $B'[0, b')$\;

$\pi(\sigma) \gets$ number of $\sigma$ in $B'[b',e')$, for each $\sigma \in \Sigma \cup \{\$\}$\;

\tcp{We check if the extension pair is terminal}
\If{ $\pi[ \$ ] > 0$ } {
      $b_{t}' \gets \Pi[ \$ ]$+1\;
      $e_{t}' \gets b_{t}' + \pi[ \$ ]$\;
      $q_{t}' \gets [b_{t}',e_{t}')$\;
\tcp{We return an arc set of the overlap graph}
      \Return $A(q_{t}', q_{2}')$\;
  }
\Else {
 \ForEach{$\sigma \in \Sigma$}{
    $b'_e \gets C[ \sigma ] + \Pi[ \sigma ]+1$\;
    $e'_e \gets b'_e + \pi[ \sigma ]$\;
      $q_{e}' \gets [b_{e}',e_{e}')$\;
    Append $(q_{t}', q_{2}')$ to $F(\sigma)$\;
      \Return an empty arc set\;

  }
}

$\Pi \gets \Pi + \pi$

\caption{extendExtensionPair($q_{1}'$, $q_{2}'$, $F$)}
\label{alg:processExtPair}
\end{algorithm}

\section{On the complexity}





We recall that the input data consist of a set of  $R$ of $n$ reads of maximum
length $l$, and that $N$ is the total length of all input reads.
% and let us recall that $n$ is the total number of reads ($R$)
% The time complexity for building the BWT of the collection $R$ of reads
% is the one reported in the paper ((Lightweight BWT Construction for Very
% Large String Collections)\cite{}).
% More precisely, using BCR we can build the data structure in
% $O(l\times sort(n))$ whereas using BCRext the time complexity
% is $O(ln)$.
Observe that the \emph{buildGraph} procedure consists of  at most $l$ iterations.
In each iteration (i) we extend the intervals in the lists $\mathcal{Q_{j}}$ (which is an
adaptation of the BCR algorithm~\cite{Bauer2013}) and $\mathcal P$, (ii) we sort
$\mathcal P$, and (iii) we extend the extension pairs in $\mathcal{E_{j}}$.

Each extension of an interval or of an extension pair consists of a linear scan
of the lists and some $O(1)$-time operations on each element of the list.
Since, for each list, the input and the output intervals are disjoint, there can
be at most $N$ intervals in each list, therefore $O(N)$ is also the time spent
at each iteration.


\begin{comment}
times the procedures ExtendQonLeft and ExtendQonRight.
In fact, those two procedures
%the procedures ExtendQonLeft and ExtendQonRight
apply the backward extension to every $Q$-interval in $\mathcal{Q}_j$ at
most $l$ times, since there can't be any interval related to a pattern
$Q$ such that $|Q| > l$. % $Q$-interval has length at least $\tau$.

On the other hand the procedures ExtendQonLeft and ExtendQonRight aim to
build all the possible $Q$-intervals linked to the maximal overlaps
between the reads (the irreducible edges of the String Graph) by means of
an incremental approach that rely on the computation of the
$Q$-intervals during the previous steps.

Thus, the time complexity of these procedures may be measured by
considering the total number of possible $Q$-intervals that we have to
compute for every length of $Q$ (more precisely for $|Q| \in \{1 \dots
l\}$).
% the fact that after the total number of iterations they have extended
% at most all possible $Q$-intervals in the data set $R$ for $|Q| \in
% \{1 \dots l\}$.

Then the total  complexity of ExtendQonLeft and ExtendQonRight is at
most $O(N_Q)$, where $N_Q$ is the total number of
$Q$-intervals (i.e. distinct substrings of $R$).

An estimate of $N_Q$ can be given as $O(l \times n)$ (let us recall that
the number of distinct strings in a word on length $x$ is  at most $x$
((Trovare articolo e citare)\cite{})).
\end{comment}

Note that during the iteration step the \emph{buildGraph} procedure has also to
sort $\mathcal{P}$. Since each interval must be sorted by its beginning (\ie a
single number), in order to sort  $\mathcal{P}$ we have to sort $n$ integers,
which can be done in $sort(n)$ time.

Since there are at most $l$ iterations, the overall time complexity is
$O(l (n + sort(n))=O(l \cdot sort(n))$.
% of BWT in external memory and thus an estimate of the time over real
% data can be ....



\section{Conclusion and future work}
%The above notion of String Graph applies to assembling RNA-seq data into
%Splicing Graph~\cite{Beretta2013}.
%When assembling RNA-seq data with an overlap graph, the final output
%graph is not a line path, since the graph should represent the
%alternative transcripts derived by the alternative splicing on the gene
%structure.
%In fact, RNA-seq data are classified into \emph{spliced} and
%\emph{unspliced} reads.
%Unspliced reads are reads that cover regions that are either coding or
%not coding ones, i.e. they are absent or present in a transcript
%sequence.
%On the contrary a spliced read $r$ contains of at least two consecutive
%substrings $r_i $, $r_{i+1}$ such that there exists at least two
%transcripts where $r_i$ is present in a transcript sequence together
%with $r_{i+1}$ while $r_i$ is present and $r_{i+1}$ is absent in the
%transcript sequence.
%
%In order to include this case we give a more general notion of
%irreducible edges of an overlap graph.
%Under the assumption that a read is not included in another one, the
%transitive reduction of an overlap graph, by which we get a String Graph
%from an overlap graph, corresponds to require that the set of reads that
%extends a read $r$ is irreducible.
%This definition includes also the case of Splicing Graphs...

% <<<<<<< HEAD
% In this paper we propose an algorithm for building a string graph from a set $R$ of reads in secondary memory using a lightweight BWT construction, while retaining
% a time complexity that  is similar to the one for building the BWT.
% Now, under the assumption that the string graph is a line graph  (as in the genomic case), we can build such a graph by directly computing only irreducible edges, otherwise
% we take an overlap graph in main memory and then we apply transitive reduction to obtain the string graph.
% An interesting open problem would be to use properties of $Q$-intervals to find irreducible edges while extending $Q$-intervals in the general case of input reads producing a string graph that is not necessarily a line graph, as in the transcriptomic application of the string graph to build a splicing-graph (\cite{Berettaet al}.
We have  proposed an external memory algorithm for building an overlap  graph from a set $R$ of reads.
Moreover we have shown that our approach is efficient, as the time complexity of
the whole algorithm is not significantly larger that that of a necessary
procedure, that is the lightweight BWT construction.

There are at least two important directions for further research; the first is
to devise an external memory algorithm to directly compute the string graph,
sidestepping the need for computing the overlap graph, the second is to
better understand the differences between string graphs associated to genomic
reads and to RNA-seqs, since in the latter context graph models of gene structure
have been recently proposed, such as the splicing graph~\cite{Beretta2013}.

\bibliographystyle{splncs03}
\bibliography{biblioBWTpaper}


\end{document}

%  LocalWords:  transcriptomic novo NGS Bruijn BWT iff unspliced
